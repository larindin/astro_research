\ProvidesFile{thesis.tex}[2025-01-15 PurdueThesis thesis.tex file]

%  The home page for the PurdueThesis software is
%      https://engineering.purdue.edu/~mark/PurdueThesis/
%
%  Be sure to sign up for the PurdueThesis mailing list at
%      https://engineering.purdue.edu/ECN/mailman/listinfo/purduethesis-list
%  so you learn of new versions of this software.  You must be on that
%  mailing list to receive help with this software.
%
%  This is the template root file for an example thesis (for master's
%  degree) or dissertation (for a Ph.D.).  From now on "thesis" will
%  refer to both of these unless stated otherwise.
%
%  LaTeX systems include auxiliary programs to do bibliographies,
%  indexes, etc.  The latexmk program runs the fewest programs needed
%  to update your thesis.
%
%  On Overleaf (run LaTeX on the web) clicking 'Recompile' will recompile
%  your thesis.
%
%  Use this command on Linux to do all the steps needed to compile your thesis.
%    For Mark Senn:
%      Use
%          latexmk -e '$biber="biber --output-safechars"' -f -g -lualatex --shell-escape thesis
%      to get extra debugging information printed.  Sometimes the `-f -g' can 
%      be deleted to run faster.
%    For everyone else:
%      Use
%          latexmk -e '$biber="biber --output-safechars"' -f -g -lualatex thesis
%      Sometimes the `-f -g' can be deleted to run faster.
%  Here is how that command works:
%      latexmk
%          The latexmk program figures out how to compile
%          your thesis in the quickest way.
%      -e '$biber="biber --output-safechars"'
%          Add the '--output-safechars' option to biber,
%          the command that makes your references.  This
%          command makes accented characters in your references
%          work ok.
%      -f
%          Force latexmk to process your entire thesis, even
%          if it contains errors.
%      -g
%          Force latexmk to process document fully, even under situations
%          where latexmk would normally decide that no changes in the
%          source files have occurred since the previous run.  This  option
%          is  useful, for example, if you change some options and wish to
%          reprocess the files.
%      -lualatex
%          The -lualatex option process your thesis using the LuaLaTeX
%          version of LaTeX.  LuaLaTeX has the Lua programmng language
%          added to make programming LaTeX easier.  You won't need to
%          learn Lua or do any LaTeX programming for your thesis.
%      --shell-escape
%          The --shell-escape option allows you to run external programs
%          from inside your thesis.
%      thesis
%          Process your thesis.tex file.
%
%
%  NOTE TO SELF
%      To count the number of references see
%          https://tex.stackexchange.com/questions/66829/count-number-of-references-using-biblate
%      See set the \labelnumberwidth see page 316 of
%          https://ctan.math.illinois.edu/macros/latex/contrib/biblatex/doc/biblatex.pdf
%      
%
%  PROGRAM                                       BIBLIOGRAPHY    PurdueThesis.cls    thesis.tex   WORKS
%                                                STYLE           RCS rev             RCS rev
%  Mathematics                                   apa             1.249               1.58         yes
%  Mathematics                                   ieee            1.250               1.59         yes
%  Electrical and Computer Engineering           ieee            1.250               1.60         yes
%  Earth, Atmospheric, and Planetary Sciences    apa             1.251               1.61         yes
%  Mathematics                                   apa             1.252               1.62         yes
%  Technology Leadership and Innovation          apa             1.253               1.63         yes
%  Mathematics                                   numeric         ????                ????         ????
%  Earth, Atmospheric, and Planetary Sciences    apa             ????                ????         ????
%

\newcommand{\ZZauthor}{Darin Lin}
\newcommand{\ZZcampus}{West Lafayette}
\newcommand{\ZZdegree}{Master of Science in Aeronautics and Astronautics}
\newcommand{\ZZdocument}{A Thesis}
\newcommand{\ZZgraduation}{December 2025}
\newcommand{\ZZinstitution}{Purdue University}
\newcommand{\ZZprogram}{Aeronautics and Astronautics}

\newcommand{\ZZtitle}{This is the Title}

\newcommand{\ZZshowcolophon}{false}
\newcommand{\ZZshowdiagonalline}{false}
\newcommand{\ZZshowgridlines}{false}
\newcommand{\ZZshowmarginlines}{false}
\newcommand{\ZZshowtimestamp}{false}
\newcommand{\ZZshowtodonotes}{false}

% Mark Senn uses an "optional-debugging-code.tex file" but does not
% distribute it.  The following line won't do anything if you don't
% have an optional-debugging-code.tex file so you can leave it the
% way it is.
\InputIfFileExists{optional-debugging-code.tex}{}{}

% The \includeonly command can be used to only include some
% files that have \include commands below.  This is handy
% to only include some files so your document will LaTeX
% faster or if you're trying to narrow down where an error
% occurs.  You can use
%   \includeonly{ch-introduction}
% to only include ch-introduction.tex, or
%   \includeonly{ch-introduction,ap-about-appendices}
% to include ch-introduction.tex and ap-about-appendices.tex.
% More files can be added---just put ',' between the names.
% Comment out the following line before submitting the
% final copy of your thesis.
%\includeonly{ch-introduction,ap-about-appendices}


\documentclass{PurdueThesis}

\newcommand{\ZZatinformation}{}

\graphicspath{{graphics/}}
\makeatletter
  \def\input@path{{misc}{packages}}
\makeatother

\ConfigureBibliography

% For \bm.
% The bm package was last updated on 2022-01-05.
\usepackage{bm}


% Define
%    \VerbatimInput[options]{filename}
%    \begin{VerbatimOut}{filename} ... \end{VerbatimOut}.
\usepackage{fancyvrb}
  % So '|verbatim|' will print 'verbatim' in a typewriter font.
  % If you don't want this, put a '%' before the next line.
  \DefineShortVerb{\|}

\usepackage{listings}

% Include XMP data in a PDF document.
% See
%   http://mirrors.ibiblio.org/CTAN/macros/latex/contrib/hyperxmp/hyperxmp.pdf
% for more information including all the possible fields that can be set.
\usepackage{hyperxmp}
  \hypersetup{
    pdfauthor    = {Mark Senn},
    pdfcopyright = {Copyright \copyright\ 2024 by Mark Senn.  All rights reserved.},
    % Use `yyyy-mm' format where `yyyy' is year and `mm' is month.
    pdfdate      = {2025-05},
    pdfkeywords  = {LaTeX; Purdue University; PurdueThesis},
    pdflang      = {en},
    pdfpublisher = {Purdue University},
    pdfsubject   = {%
                     PurdueThesis is a LaTeX document class used for
                     master's bypass reports,
                     master's theses,
                     PhD dissertations,
                     and PhD preliminary reports.
                     This template demonstrates how to use PurdueThesis.%
                   },
    pdftitle     = {This is the Title},
  }

\usepackage{mathtools}
\usepackage{multicol}
\usepackage{pa-logos}
\def\pa{\rotatebox[origin=c]{14}{\partial}}
\def\Fourier{\mathcal{F}}
\def\Laplace{\mathcal{L}}
\usepackage{pa-repeat}
\usepackage{placeins}

% Needed for chapter "Graphics", section "TikZ and PGF".
\usepackage{tikz}
  % Needed to customize arrows.
  \usetikzlibrary{arrows.meta}
  % For electrical diagrams.
  % Uses the TikZ package.
  % The circuitikz name is short for "circuit TikZ".
  \usepackage{circuitikz}
  %
  \usepackage{menukeys}
  %
  % Needed for 3D TikZ stuff.
  \usetikzlibrary{3d}
  %
  % Needed for pa-typographic-conventions package.
  \usetikzlibrary{calc,shadows,shapes.misc,shapes.symbols}
  %
  % Needed for putting text along a path.
  \usetikzlibrary{decorations.text}
  %
  % Draw TikZ decorations.
  % Needed for at least the Kalman filter system model graphic.
  \usetikzlibrary{decorations.pathmorphing} % noisy shapes
  %
  % Fit shapes to coordinates.
  % Needed for at least the Kalman filter system model graphic.
  \usetikzlibrary{fit}
  %
  % Draw the background after the foreground.
  \usetikzlibrary{backgrounds}	% drawing the background after the foreground

% The vertical space between a table heading and the table contents
% in a tabular environment.
\newcommand{\tabularspace}{\noalign{\vspace*{2pt}}}

% For \sfrac, used to do slanted fractions, similar to, e.g., 1/2,
% but 1 is small and high and 2 is small and low.
\usepackage{xfrac}


% Define \I.
% \I1 does \indent once, \I2 does \indent twice, etc.
\newcommand{\I}[1]{\MyRepeat{\indent}{#1}}

% Define \MyI.
% Typeset my input.
\long\def\MyI#1%
  {%
    {%
      \fontsize{8}{10}\tt
      \VerbatimInput
        [
          firstnumber = 1,
          numbers     = left,
          xleftmargin = 0.33in,
        ]%
        {#1}
    }%
  }

% Define \MyIO.
% Typeset my input and output.
% The input will all be on the same page.
% The output may be split over multiple pages.
\newcommand{\MyIO}
  {%
    \input{z.out}

    {%
      \fontsize{8}{10}\tt
      \VerbatimInput
        [
          firstnumber = 1,
          numbers     = left,
          xleftmargin = 0.33in,
        ]
        {z.out}
    }
    \FloatBarrier
  }

% Define \NL (newline) so LaTeX goes to the next output line.
% Just doing \\ complains
%     ! LaTeX Error: There's no line here to end.
% \mbox{} is an empty math box.
\newcommand{\NL}{\mbox{}\\}

% In this document I use in-line tables a lot.
% These are tables that are put in the document
% where I want them to appear and they don't
% use \begin{table} ... \end{table}
\newenvironment{inlinetable}
  {%
    \begingroup
      \singlespace
      \mbox{}\\[-9pt]%
      \noindent
      \hspace*{2\parindent}%
      \ignorespaces
  }
  {%
      \mbox{}\\
    \endgroup
  }

\listfiles
\usepackage{pa-typographic-conventions}

\begin{document}

\makeatletter
\renewcommand{\ZZAppendixName}{APPENDIX}
%%%% \renewcommand{\chaptername}{CHAPTER}
\def\@@makechapterhead#1{\uppercase{\@chapapp~\thechapter. #1}}

\setcounter{tocdepth}{3}

\maketitle

\ProvidesFile{ch-front.tex}[2024-09-12 front matter chapter]
%
%  This is the ``front matter'' for the thesis.
%
%  REFERENCES
%
%    TCMOS17
%      The Chicago Manual of Style Online, 17th edition.
%      https://www.chicagomanualofstyle.org/home.html
%      retrieved on 2020-02-29
%
%    TEMPL
%      Thesis and Disertation Office Templates.
%      https://www.purdue.edu/gradschool/research/thesis/templates.html
%      retrieved on 2020-02-29
%
%    WNNCD
%    Webster's Ninth New Collegiate Dictionary.
%

%
%   Only Purdue University uses this page
%
%   Comment out \begin{statement} through \end{statement}
%   if you are not at Purdue University.
%
% Statement of Thesis/Dissertation Approval Page
% This page is REQUIRED.  The page should be numbered "2"
% and should NOT be listed in your TABLE OF CONTENTS.
\begin{statement}
  % Delete or add \entry commands as needed for all committe members.
  \entry{Dr.~Kenshiro Oguri, Chair}{School of Aeronautics and Astronautics}
  \entry{Dr.~Inseok Hwang}{School of Aeronautics and Astronautics}
  \entry{Dr.~Keith LeGrand}{School of Aeronautics and Astronautics}
  % There should be one \approvedby command containing the
  % "FORM 9 THESIS FORM HEAD NAME HERE" (from TEMPL, retrieved on 2020-03-01).
  \approvedby{Dr.~Buck Doe}
\end{statement}

% Dedication page is optional.
% A name and often a message in tribute to a person or cause.
% References: WEB9 332.
\begin{dedication}
  To graduate students
\end{dedication}

% Acknowledgements page is optional but most theses include
% a brief statement of appreciation or recognition of special
% assistance.
\begin{acknowledgments}
  Purdue University's Engineering Computer Network
  (now part of Purdue IT)
  and Graduate School helped fund \PurdueThesisLogo\ development.
\end{acknowledgments}

% The preface is optional.
% References: TCMOS17 1.49, WEB9 927.
\begin{preface}
  This is the preface.
\end{preface}

% The Table of Contents is required.
% The Table of Contents will be automatically created for you
% using information you supply in
%     \chapter
%     \section
%     \subsection
%     \subsubsection
%     commands.
\pdfbookmark{TABLE OF CONTENTS}{Contents}
\tableofcontents

% If your thesis has tables, a list of tables is required.
% The List of Tables will be automatically created for you using
% information you supply in
%     \begin{table} ... \end{table}
% environments.
\listoftables

% If your thesis has figures, a list of figures is required.
% The List of Figures will be automatically created for you using
% information you supply in
%     \begin{figure} ... \end{figure}
% environments.
\listoffigures

% If your thesis has listings, a list of listings is required.
% The List of Listings will be automatically created for you using
% information you supply in
%     \begin{ZZlisting} ... \end{ZZlisting}
% environments.
\ZZlistoflistings

% List of Symbols is optional.
\begin{symbols}
  \(m\)& mass\cr
  \(v\)& velocity\cr
\end{symbols}

% List of Abbreviations is optional.
\begin{abbreviations}
  abbr& abbreviation\cr
  bcf& billion cubic feet\cr
  BMOC& big man on campus\cr
\end{abbreviations}

\begin{abstract}
  Tracking maneuvering cislunar spacecraft is a difficult task due to the highly nonlinear dynamical environment, great distances, and frequent observation gaps. If optimal control is assumed, it is possible to predict the future control of a spacecraft given observations of the start of a maneuver. This idea is applied to construct an optimal control interacting multiple model estimator (OCIMM) by including the costate as an estimation variable. The OCIMM can significantly reduce the mean absolute estimation error compared to a traditional IMM during observation gaps and periods of rapidly changing control through its ability to predict the future control. 
\end{abstract}


%
% Put chapter \include commands here.
%

% 'Important---Read This First' chapter.
\ProvidesFile{ch-important.tex}[2024-07-12 important chapter]

\chapter{IMPORTANT---READ THIS FIRST}

Be sure to sign up for the
\href{https://engineering.purdue.edu/ECN/mailman/listinfo/purduethesis-list}{\PurdueThesisLogo\ mailing list}%
\cite{PurdueThesis-mailing-list}
to learn of important changes to
or get help with \PurdueThesisLogo.

I suggest you do not make any changes
to the |PurdueThesis.cls| file.
Put any changes in the |thesis.tex| file if you can.
That way you will not need to add your customizations
when a new version of |PurdueThesis.cls| is released.


\makeatletter  % commented out on 2022-01-26
  \defbibenvironment{bibliography}
    {%
      \list
        {%
          \printtext[labelnumberwidth]%
          {%
            \printfield{prefixnumber}%
            \printfield{labelnumber}%
          }%
        }%
        {%
          \setlength{\bibhang}{1in} %%%%% was 0pt
          \setlength{\itemindent}{1in}%  -\leftmargin} %%%%% was 0pt
          \setlength{\itemsep}{\bibitemsep}%
          \setlength{\leftmargin}{0pt}%  .22in} % 0.42in}
          \setlength{\parsep}{\bibparsep}%
          \setlength{\rightmargin}{0.33in}%
        }%
    }
    {\endlist}
    {\item}
\makeatother  % commented out on 2022-01-26

% \immediate\setlength{\labelnumberwidth}{1.5in} %%%%% was commented out
\setlength{\labelwidth}{1.5in}

% Appendices are optional.  Not all theses contain appendices.
% An appendix is used for suppleentary illustrative material,
% original data, computer programs, and other material that is not
% necessarily appropriate for inclusion within the text of your
% thesis.
% Reference: TM2017 page 33.
%
% Use ``\appendix'' for one appendix or ``\appendices'' for more than
% one appendix.
\appendices

\ProvidesFile{ap-about-appendices.tex}[2022-10-05 about the appendicies appendix]

\begin{VerbatimOut}{z.out}
\chapter{ABOUT THE APPENDICES}

% Use single spacing in the appendices from now on to save space.
\ZZbaselinestretch{1}

\textcolor{red}{%
  \textbf{%
    These appendices are single-spaced to save space.
    Your thesis should use the default~1.5 line spacing.%
  }%
}

There are two groups of appendices.
The first group are general appendices;
the second group are domain-specific appendices.

These appendices are a series of examples.
They are a work in progress.

Each example consists of some \LaTeX\ output
followed by the corresponding input lines.
Some \LaTeX\ input lines only define things
and don't produce any output.
Each chunk in the input file begins with
\verb+\begin{VerbatimOut}{z.out}+
then has the \LaTeX\ input for the example,
% Don't literally end VerbatimOut on next line.
and ends with {\tt \char'134 end\char'173 VerbatimOut\char'175},
followed by a blank line,
followed by a line that begins with
|\My|.

\end{VerbatimOut}

\MyIO


\begin{VerbatimOut}{z.out}


\section{Paragraphs}

This is the first paragraph.
Paragraphs are separated by blank lines.

This is the second paragraph.


\section{Section Heading}

This is a sentence.
This is a sentence.
This is a sentence.
This is a sentence.
This is a sentence.


\subsection{Subsection heading}

This is a sentence.
This is a sentence.
This is a sentence.
This is a sentence.
This is a sentence.


\subsubsection{Subsubsection heading}

This is a sentence.
This is a sentence.
This is a sentence.
This is a sentence.
This is a sentence.
\end{VerbatimOut}

\MyIO



\begin{VerbatimOut}{z.out}


\section{Text math}

If items in a list are narrow like these Greek characters,\\
    \I2 \verb+$\alpha$, $\beta$, and $\gamma$+\\
I'd input the line like this\\
    \I2 \verb+$\alpha$,~$\beta$, and~$\gamma$+\\
where the \verb+~+ is a tie
that ties together what's before and after it on the same line of the output
\cite[page~92]{knuth2012}.

This text is the correct length to show what happens with and without ties:
$\alpha$,
$\beta$,
and $\gamma$.
See how the line gets split
and the~$\gamma$ is at the beginning of the line?

This text is the correct length to show what happens with and without ties:
$\alpha$,~$\beta$,
and~$\gamma$.
See how the line gets compressed a little bit so the~$\gamma$
is not at the beginning of the line?
\end{VerbatimOut}

\MyIO


% My filename conventions:
%     FILE THAT START WITH    ARE
%     ap-                     appendices
%     ch-                     chapters
%     gr-                     graphics
%     pa-                     packages
%     z                       temporary files
\end{document}
