\ProvidesFile{ap-mathematics.tex}[2023-09-01 mathematics appendix]

\begin{VerbatimOut}{z.out}
\chapter{MATHEMATICS}
\ix{mathematics//Mathematics appendix}

\PurdueThesisLogo\ loads the \AMSmathLogo\ package
\cite{amslatex3project2019}
to do mathematics.
\end{VerbatimOut}

\MyIO


\begin{VerbatimOut}{z.out}
There are two types of mathematics in \LaTeXLogo.
Text math is interspersed with text.
For example,
this is text math: \(a = b + c\).
Display math is not interspersed with text.
For example,
this is display math:
\begin{equation}
  a = b + c
\end{equation}
\end{VerbatimOut}

\MyIO


\begin{VerbatimOut}{z.out}


\section{Text Math}

Use |\(|
to start text math
and |\)|
to end text math.
Some people use |$|
to start and end text math---I don't
recommend that because \LaTeXLogo\ can give better error messages
if you use |\(|
and |\)|
\cite{meckes2010}.
\end{VerbatimOut}

\MyIO


\begin{VerbatimOut}{z.out}


\section{Display Math}

Use one of the below environments to start and end display math.
Some people use |$$|
to start and end displayed math
but \LaTeXLogo\ doesn't officially support |$$|
\cite{alpert2010}.


\end{VerbatimOut}

\MyIO


\begin{VerbatimOut}{z.out}


\subsection{Displayed Equations}

Do not use
|$$|
to start or end displayed math like \TeXLogo\ uses
\cite{gratzer2016}.
\todoerror{add page number to reference}

The \AMSmathLogo\ package provides a number
of additional displayed equation structures
beyond the ones provided in basic \LaTeX.
The augmented set includes
\cite{amslatex3project2019b}:

\hbox to\hsize{%
  \hss
  \begin{tabular}{@{}ll@{}}
    \toprule
    \bfseries Environment& \bfseries Used for\\
    \midrule
    \tt equation& used for single equations\\
    \tt multline& split single equations over multiple lines\\
    \tt gather& collect but do not align multiple equations\\
    \tt align& align multiple equations\\
    \tt alignat& aligns multiple equations at multiple places\\
    \tt flalign& aligns multiple equations at multiple places on full length lines\\
    \tt split& split a single equation over multiple lines\\
    \bottomrule
  \end{tabular}%
  \hss
}

All but
|split|
can be followed by
|*|
to not number equations.
\end{VerbatimOut}

\MyIO


\begin{VerbatimOut}{z.out}

\subsubsection{\texttt{equation} environment}

The
|equation|
environment is used for single equations.

\begin{equation}
  E = mc^2
\end{equation}
\end{VerbatimOut}

\MyIO


\begin{VerbatimOut}{z.out}

The
|equation*|
environment does single, unnumbered equations.

\begin{equation*}
  a = b_0c + \frac12 de^2 + {\textstyle \frac12} fg^2
    + h_1 + h_2 + \cdots + h_n
    \qquad \text{for \(c \ne d\) and \(g < \infty\)}
\end{equation*}
\end{VerbatimOut}

\MyIO


\begin{VerbatimOut}{z.out}

\textcite{greene-2021-03-14}
wrote
% \begin{quotation}
  For
  \href{https://twitter.com/hashtag/PiDay?src=hashtag\_click\#PiDay}{\#PiDay},
  one of the coolest formulae for today's honoree:
  \[
    \frac 1\pi
    =
    \frac {\sqrt8} {9801}
    \sum_{n=0}^\infty
    \frac  {(4n!) (1103+26390n)}  {(n!)^4 396^{4n}}
  \]
% \end{quotation}
\end{VerbatimOut}

\MyIO


\begin{VerbatimOut}{z.out}

The formula for Bekenstein-Hawking entropy:

\begin{equation*}
  S_\text{BH}
  =
  \frac A {4L_P^2}
  = \frac {c^3A} {4G\hbar}
\end{equation*}
\end{VerbatimOut}

\MyIO


\begin{VerbatimOut}{z.out}

Type in the math and let \LaTeX\ worry about the spacing.
You only need to do fine tuning by hand if it looks bad.

Another
|equation*|
environment,
note the spacing before the large close parenthesis:

\begin{equation*}
  \frac ab
    = ab^{-1}
    % Parens are the wrong size.
    = (\sqrt{\frac ab})^2
    % Parens are the right size but closing paren is too close to radical.
    = \left( \sqrt\frac ab \right)^2
    % Parens are right size but a negative thin space puts closing paren on top of radical.
    = \left( \sqrt\frac ab \!\right)^2
    % Parens are right size but a thin space puts closing paren too close to radical.
    = \left( \sqrt\frac ab \,\right)^2
    % Parens are right size but a medium space puts closing paren too close to radical.
    = \left( \sqrt\frac ab \:\right)^2
    % Parens are right size and I think a thick space looks the best.
    = \left( \sqrt\frac ab \;\right)^2
\end{equation*}
\end{VerbatimOut}

\MyIO


\begin{VerbatimOut}{z.out}

\begin{equation*}
  (\cos x)^2 + (\sin x)^2 = \cos^2 x + \sin^2 x = 1
\end{equation*}
\end{VerbatimOut}

\MyIO


\begin{VerbatimOut}{z.out}

\begin{equation}
  x \mod 2 =
  \begin{cases}
    0& \text{if \(x\) is even}\\
    1& \text{if \(x\) is odd}\\
  \end{cases}
\end{equation}
\end{VerbatimOut}

\MyIO


\begin{VerbatimOut}{z.out}

The first six derivatives of distance are velocity, acceleration, jerk, snap, crackle,
and pop
\cite{reid2013}.
\ix{distance//velocity//acceleration//jerk//snap//crackle//pop}

\begin{equation}
  % Every array element should be in \displaystyle (a big font).
  \AtBeginEnvironment{array}{\everymath{\displaystyle}}
  % Set space between columns to zero, use {} = ... to add a little space before the = "by hand".
  \arraycolsep = 0pt
  \text{distance derivitives} = \left\{\ %
    \begin{array}{llllllll}
      % I'm formatting the first 4 lines different from the last 3 so this will fit on one page.
      x&      {}=\text{distance}&     {}=vt\\[2pt]
      v&      {}=\text{velocity}&     {}=\frac{\di x}{\di t}\\[9pt]
      a&      {}=\text{acceleration}& {}=\frac{\di v}{\di t}& {}=\frac{\di^2x}{\di t^2}\\[9pt]
      \mit j& {}=\text{jerk}&         {}=\frac{\di a}{\di t}& {}=\frac{\di^2v}{\di t^2}&
        {}=\frac{\di^3x}{\di t^3}\\[9pt]
      s
        & {}=\text{snap}
        & {}=\frac{\di \mit j}{\di t}
        & {}=\frac{\di^2a}{\di t^2}
        & {}=\frac{\di^3v}{\di t^3}
        & {}=\frac{\di^4x}{\di t^4}\\[9pt]
      c
        & {}=\text{crackle}
        & {}=\frac{\di s}{\di t}
        & {}=\frac{\di^2\mit j}{\di t^2}
        & {}=\frac{\di^3a}{\di t^3}
        & {}=\frac{\di^4v}{\di t^4}
        & {}=\frac{\di^5x}{\di t^5}\\[9pt]
      p
        & {}=\text{pop}
        & {}=\frac{\di c}{\di t}
        & {}=\frac{\di^2s}{\di t^2}
        & {}=\frac{\di^3\mit j}{\di t^3}
        & {}=\frac{\di^4a}{\di t^4}
        & {}=\frac{\di^5v}{\di t^5}
        & {}=\frac{\di^6x}{\di t^6}
    \end{array}
  \right.
\end{equation}
\end{VerbatimOut}

\MyIO


\begin{VerbatimOut}{z.out}

\subsubsection{\texttt{multline} environment}

The
|multline|
environment is used
to split single equations over multiple lines.

\begin{multline}
  S = a + b + c + d + e + f + g + h + i + j\\
  + k + l + m + n + o + p\\
  + q + r + s + t + u + v + w + x + y + z
\end{multline}
\end{VerbatimOut}

\MyIO


\begin{VerbatimOut}{z.out}

\begin{multline}
  S = a + b + c + d + e\\
  + f + g + h + i + j\\
  + k + l + m + n + o\\
  + p + q + r + s + t\\
  + u + v + w + x + y\\
  + z
\end{multline}
\end{VerbatimOut}

\MyIO


\begin{VerbatimOut}{z.out}

% Calculate width of space before equation plus equation number.
% (All digits are the same width.)
\newdimen{\tdimen}
\settowidth{\tdimen}{\kern\multlinetaggap (L.5)}
\begin{multline}
  S = a + b + c + d + e\\
  \makebox[\textwidth]{\hfill \(+ f + g + h + i + j\)\hfill\hfill\hfill\hfill\kern\tdimen}\\
  \makebox[\textwidth]{\hfill\hfill\({} + k + l + m + n + o\)\hfill\hfill\hfill\kern\tdimen}\\
  \makebox[\textwidth]{\hfill\hfill\hfill\({} + p + q + r + s + t\)\hfill\hfill\kern\tdimen}\\
  \makebox[\textwidth]{\hfill\hfill\hfill\hfill\({} + u + v + w + x + y\)\hfill\kern\tdimen}\\
  + z
\end{multline}
\end{VerbatimOut}

\MyIO


\begin{VerbatimOut}{z.out}

\subsubsection{\texttt{gather} environment}

The
|gather|
environment collects but does not align multiple equations.

\begin{gather}
  a = b + c + d + e + f + g + h + i + j + k + l\\
  m = n + o + p + q + r + s + t + u + v + w + x + y + z
\end{gather}
\end{VerbatimOut}

\MyIO


\begin{VerbatimOut}{z.out}

\begin{gather}
  a = b + c + d + e + f + g + h + i + j + k + l\notag\\
  m = n + o + p + q + r + s + t + u + v + w + x + y + z
\end{gather}
\end{VerbatimOut}

\MyIO


\begin{VerbatimOut}{z.out}

\begin{gather*}
  \alpha = \beta + \gamma + \delta + \eta\\
  \theta = \iota + \kappa + \lambda + \mu + \nu + \rho + \tau
\end{gather*}
\end{VerbatimOut}

\MyIO


\begin{VerbatimOut}{z.out}

\begin{gather}
  x_\text{min} + x_\text{max} \le \sum_{i=1}^n x_i\\
  x_\text{min} + x_\text{max}
    = \sum_{i=1}^n x_i - \sum_{i=2}^{n-1} x_i\quad\text{if \(x\) is sorted}\\
  x_\text{min} \le \left(\sum_{i=1}^n x_i\right) / n
\end{gather}
\end{VerbatimOut}

\MyIO


\begin{VerbatimOut}{z.out}

\subsubsection{\texttt{align} environment}

The
|align|
environment aligns multiple equations.

\begin{align}
  a &= b + c + d\\
  e &= f + g + h + i + j
\end{align}
\end{VerbatimOut}

\MyIO


\begin{VerbatimOut}{z.out}

\begin{align}
  x = \frac{-b \pm \sqrt{b^2-4ac}}{2a}\notag\\
  % Put a thin space before the b^2 to improve the appearance.
  x = \frac{-b \pm \sqrt{\,b^2-4ac}}{2a}
\end{align}
\end{VerbatimOut}
\ix{align environment}
\index{\verb+\begin{align}+}
\ix{thin space}
\index{\verb+\,+}

\MyIO


\begin{VerbatimOut}{z.out}

Quadratic formula proof
\cite{khan2018}:
\ix{quadratic formula}

% The align environment requires the amsmath package.
% Use \addtolength{\jot}{6pt} to increase the space between rows in an amsmath multi-line math formula.
% That's not done here so everything will fit on one page.
\begin{align}
  ax^2 + bx + c &= 0\\
  ax^2 + bx &= -c\notag\\
  % The "\," adds a thinspace of horizontal space.
  x^2 + \frac ba\,x &= -\frac ca\notag\\
  x^2 + \frac ba\,x + \frac{b^2}{4a^2} &= \frac{b^2}{4a^2} - \frac ca\notag\\
  \left(x + \frac b{2a}\right)^2 &= \frac{b^2}{4a^2} - \frac ca\notag\\
  \left(x + \frac b{2a}\right)^2 &= \frac{b^2}{4a^2} - \frac{4ac}{4a^2}\notag\\
  \left(x + \frac b{2a}\right)^2 &= \frac{b^2-4ac}{4a^2}\notag\\
  \sqrt{\left(x + \frac b{2a}\right)^2}
    &= \sqrt{\left(\frac{b^2-4ac}{4a^2}\right)}\notag\\
  x + \frac b{2a} &= \pm \frac{\sqrt{\,b^2-4ac}}{\sqrt{4a^2}}\notag\\
  x + \frac b{2a} &= \pm \frac{\sqrt{\,b^2-4ac}}{2a}\notag\\
  x &= - \frac b{2a} \pm \frac{\sqrt{\,b^2-4ac}}{2a}\notag\\
  x &= \frac{-b \pm \sqrt{\,b^2-4ac}}{2a}
\end{align}
\end{VerbatimOut}

\MyIO


\begin{VerbatimOut}{z.out}

\subsubsection{\texttt{alignat} environment}
\index{\verb+\begin{aligo}+@\verb+\begin{alignat}+}
\ix{alignat environment}

The
|alignat|
environment aligns multiple equations at multiple places.
\begin{alignat}{2}
  a &= b& \qquad\qquad& \text{set \(a\)}\\
  c &= d& &             \text{you guessed it, set \(c\)}\notag\\
  g &= h& &             \text{and finally, set \(g\)}
\end{alignat}
\index{\verb+\begin{aligo}+@\verb+\begin{alignat}+}
\ix{alignat environment}

I like to align input columns on the input if possible
and will sometimes use windows over~250 characters wide to do so.
If that won't work I sometimes do,
for example,
\begin{alignat}{2}
  a
    &= b
    & \qquad\qquad
    & \text{set \(a\)}\\
  c
    &= d
    &
    &\text{you guessed it, set \(c\)}\notag\\
  g
    &= h
    &
    &\text{and finally, set \(g\)}
\end{alignat}
\index{\verb+\begin{aligo}+@\verb+\begin{alignat}+}
\ix{alignat environment}

Do whatever works best for you.

\end{VerbatimOut}

\MyIO


\begin{VerbatimOut}{z.out}

Quadratic formula proof
\cite{khan2018}:

% Make changes to \jot be local to the group that starts on the next line.
{
  % Increase distance between lines by 6pt.
  \addtolength{\jot}{6pt}
  \begin{alignat}{2}
    ax^2 + bx + c
      &= 0
      &
      &\text{subtract \(c\)}\\
    ax^2 + bx
      &= -c
      &
      &\text{divide by \(a\)}\notag\\
    % The "\," adds a thinspace of horizontal space.
    x^2 + \frac ba\,x
      &= -\frac ca
      &
      &\text{add \(\displaystyle\frac{b^2}{4a^2}\)}\notag\\
    x^2+\frac ba\,x+\frac{b^2}{4a^2}
      &= \frac{b^2}{4a^2}-\frac ca
      &
      &\text{factor left hand side}\notag\\
    \left(x+\frac b{2a}\right)^2
      &= \frac{b^2}{4a^2}-\frac ca
      &
      &\text{multiply right-most term by \(\displaystyle\frac{4a}{4a}\)}\notag\\
    \left(x + \frac b{2a}\right)^2
      &= \frac{b^2}{4a^2}-\frac{4ac}{4a^2}
      &
      &\text{use common denominator}\notag\\
    \left(x + \frac b{2a}\right)^2
      &= \frac{b^2-4ac}{4a^2}
      &
      &\text{take square root of each side}\notag\\
    \sqrt{\left(x + \frac b{2a}\right)^2}
      &= \sqrt{\left(\frac{b^2-4ac}{4a^2}\right)}
      &
      &\text{compute square root of each side}\notag\\
    x + \frac b{2a}
      &= \pm \frac{\sqrt{\,b^2-4ac}}{\sqrt{4a^2}}
      &
      &\text{simplify right hand denominator}\notag\\
    x + \frac b{2a}
      &= \pm \frac{\sqrt{\,b^2-4ac}}{2a}
      &
      &\text{subtract \(\displaystyle\frac b{2a}\) from each side}\notag\\
    x
      &= -\frac b{2a} \pm \frac{\sqrt{\,b^2-4ac}}{2a}
      &\qquad
      &\text{use common denominator}\notag\\
    x
      &= \frac{-b \pm \sqrt{\,b^2-4ac}}{2a}
  \end{alignat}
}
\end{VerbatimOut}

\MyIO


\begin{VerbatimOut}{z.out}

\index{\verb+\begin{flalign}+}
\todoindex{Verb+Begin-Ocurly-flalign-Ccurly+}
\ix{falign environment}
\subsubsection{\texttt{flalign} environment}

The
|flalign|
environment aligns multiple equations at multiple places
on full length lines.

\begin{flalign}
  a &= b&   &   & u &= v\\
  c &= d& m &= n& w &= x\notag\\
  g &= h&   &   & y &= z
\end{flalign}
\end{VerbatimOut}

\MyIO


\begin{VerbatimOut}{z.out}

\index{\verb+\begin{split}+}
\todoindex{Verb+Begin-Ocurly-split-Ccurly+}
\ix{split environment}
\subsubsection{\texttt{split} environment}

The
|split|
environment ???.
\index{\verb+\begin{split}+}
\todoindex{Verb+Begin-Ocurly-split-Ccurly+}
\ix{split environment}


\end{VerbatimOut}

\MyIO


\begin{VerbatimOut}{z.out}


\newpage
\section{Use the Following in Text or Display Math}

The following constructs can be used in text or display math.
\end{VerbatimOut}

\MyIO


\begin{VerbatimOut}{z.out}

\subsection{Breaking Lines}

This information is based on information from ISO 80000-2
\cite[page 2]{iso80000-2}:
\begin{quote}
  If an expression or equation must be split into two or more lines,
  place the line breaks immediately before one of the symbols
  \(=\),
  \(+\),
  \(-\),
  \(\times\),
  \(/\),
  \(\pm\),
  \(\mp\),
  etc.
\end{quote}

Examples:

The alternating sum of the first 20 prime numbers is
\(2 - 3 + 5 - 7 + 11 - 13 + 17 - 19\\+ 23 - 29 + 31 - 37 + 41 - 43 + 47 -53 + 59 - 61 + 67 - 71\).
\begin{align}
  s = {} & 2 - 3 + 5 - 7 + 11 - 13 + 17 - 19 + 23 - 29\nonumber\\
         & + 31 - 37 + 41 - 43 + 47 -53 + 59 - 61 + 67 - 71\nonumber
\end{align}

\end{VerbatimOut}

\MyIO


\begin{VerbatimOut}{z.out}

\subsection{Constants, etc., should be in an upright font}

This information is based on information
in ISO 80000-2
\cite[page 1]{iso80000-2}:
\begin{itemize}
  \item
    The constants \(\mit e\),~\(\mit i\),~\(\mit j\), and~\(\itpi\)
    should be typeset in an upright font as \(e\),~\(i\),~\(j\) and~\(\pi\).\\
    Example: \(\displaystyle e^{i\pi} + 1 = 0\).
  \item
    The ordinary derivative operator \(d\)
    should be typeset in an upright font as `\(\mathrm{d}\)'.\\
    Example: \(\displaystyle \int 2x\di x = x^2 + C\).
  \item
    The partial derivative operator \(\partial\)
    should be typeset in an upright font as \(\pa\).
    Example:
    \(
      \displaystyle
      \frac {\pa f} {\pa x}
      =
      f_x
      =
      \lim_{h\to 0}
      \frac {f(x + h,y) - f(x,y)} {h}
      =
      \lim\nolimits_{h\to 0}
      \frac {f(x + h,y) - f(x,y)} {h}
    \).
\end{itemize}
\end{VerbatimOut}

\MyIO


\begin{VerbatimOut}{z.out}
\newpage
The |thesis.tex| file sets up the first capability above with
\begin{verbatim}
% Follow ISO 80000-2:2019
%     o   put e, i, j, and pi in upright font automatically
%     o   use, for example, "\di x" to get "\,mathrm{d}\/x"
\usepackage{pa-mismath}
  % Put e, i, j, and pi in upright font automatically.
  % Comment the corresponding line to not put the symbol in an upright font.
  \enumber
  \inumber
  \jnumber
  \pinumber
  % With the four lines above not commented out, to typeset math italic e,
  % i, j, and pi use
  %     \mathit e
  %     \mathit i
  %     \mathit j
  %     \itpi
\end{verbatim}
\end{VerbatimOut}

\MyIO


\begin{VerbatimOut}{z.out}
\subsection{English Words}

English words in math should be in a roman font like this:\\
Let the maximum value of \(a\) be \(a_\text{max}\).\\
\(a_\text{max} \ge a_\text{min}\) should always be true.\\
The temperature in the attic is \(t_\text{attic}\).
\end{VerbatimOut}

\MyIO


\begin{VerbatimOut}{z.out}

\subsection{Functions}

Standard functions should be in a roman font.
Like this: \(\cos\theta\).
Here is a list of standard function commands:\\

% The "@{\hspace*{\parindent}}" indents the table
% the same amount as a paragraph.
\begin{tabular}{@{\hspace*{\parindent}}llll@{}}
  \verb+\arccos+& \verb+\csc+& \verb+\ker+&    \verb+\min+\\
  \verb+\arcsin+& \verb+\deg+& \verb+\lg+&     \verb+\Pr+\\
  \verb+\arctan+& \verb+\det+& \verb+\lim+&    \verb+\sec+\\
  \verb+\arg+&    \verb+\dim+& \verb+\liminf+& \verb+\sin+\\
  \verb+\cos+&    \verb+\exp+& \verb+\limsup+& \verb+\sinh+\\
  \verb+\cosh+&   \verb+\gcd+& \verb+\ln+&     \verb+\sup+\\
  \verb+\cot+&    \verb+\hom+& \verb+\log+&    \verb+\tan+\\
  \verb+\coth+&   \verb+\inf+& \verb+\max+&    \verb+\tanh+\\
\end{tabular}
\ix
{%
  arccos//arcsin//arctan//arg//cos//cosh//cot//coth%
  //csc//deg//det//dim//exp//gcd//hom//inf%
  //ker//lg//lim//liminf//limsup//ln//log//max%
  //min//Pr//sec//sin//sinh//sup//tan//tanh%
}
\index{\verb+\arccos+} \index{\verb+\arcsin+} \index{\verb+\arctan+} \index{\verb+\arg+}
\index{\verb+\cos+} \index{\verb+\cosh+} \index{\verb+\cot+} \index{\verb+\coth+}
\index{\verb+\csc+} \index{\verb+\deg+} \index{\verb+\det+} \index{\verb+\dim+}
\index{\verb+\exp+} \index{\verb+\gcd+} \index{\verb+\hom+} \index{\verb+\inf+}
\index{\verb+\ker+} \index{\verb+\lg+} \index{\verb+\lim+} \index{\verb+\liminf+}
\index{\verb+\limsup+} \index{\verb+\ln+} \index{\verb+\log+} \index{\verb+\max+}
\index{\verb+\min+} \index{\verb+\Pr+} \index{\verb+\sec+} \index{\verb+\sin+}
\index{\verb+\sinh+} \index{\verb+\sup+} \index{\verb+\tan+} \index{\verb+\tanh+}
\end{VerbatimOut}

\MyIO


\begin{VerbatimOut}{z.out}

\subsection{Matrices}

From ISO 80000-2
\cite[page 18]{iso80000-2}:
\begin{quotation}
  Matrices are usually written with boldface italic capital letters
  and their elements with italic lower case letters,
  but other typefaces may be used.
  \todoerror{Double check this quote.}
\end{quotation}

Example:
Let \(\bm{M}\) be a \(3 \times 3\) matrix:
\begin{equation*}
  \bm{M}
  =
  \left(
    \begin{array}{ccc}
      m_{1,1}& m_{1,2}& m_{1,3}\\
      m_{2,1}& m_{2,2}& m_{2,3}\\
      m_{3,1}& m_{3,2}& m_{3,3}\\
    \end{array}
  \right)
  =
  \left(
      % If your thesis.tex is set up to assume, for example, 'e' is a constant
      % and typeset in an upright font, use '\mit e' to typeset 'e' in a math
      % italic font.
    \begin{array}{ccc}
      a& b& c\\
      d& \mit e& f\\
      g& h& \mit i\\
    \end{array}
  \right)
\end{equation*}
\end{VerbatimOut}

\MyIO


\begin{VerbatimOut}{z.out}

\subsection{Sets}

% Make \newcommand local to { ... }.
{
  % \St is short for my set.
  \newcommand{\Se}[1]{\ensuremath{\mathbf #1}}
  % \St is short for suchthat.
  \newcommand{\St}{\mid}
  \noindent
  Use \(\mathbb{N}\) for the natural numbers.
  \index{natural numbers}\\
  Use \(\mathbb{R}\) for the real numbers.
  \index{real numbers}\\
  Use \(\mathbb{Z}\) for the integers.
  \index{integers}\\
  The Cartesian product\index{\verb+\times+}\index{Cartesian product}
  of \Se A and \Se B is
  \(\Se A \times \Se B = \{(a,b) \St a \in \Se A \text{ and } b \in \Se B\}\).\\
  The intersection\index{\verb+\cap+}\index{intersection}
  of \Se A and \Se B is
  \(\Se A \cap \Se B = \{ x \St x \in \Se A \text{ and } x \in \Se B\}\).\\
  The union\index{\verb+\cup+}\index{union}
  of \Se A and \Se B is
  \(\Se A \cup \Se B = \{ x \St x \in \Se A \text{ and } x \in \Se B\}\).
}
\end{VerbatimOut}

\MyIO


\begin{VerbatimOut}{z.out}


\section{Theorem-like environments}

The |ntheorem| package\index{ntheorem package}
is used instead of |amsthm|\index{amsthm package}
so the counter
for a theorem-like construct
can be overridden by the user.
See
\cite[pages 4--5]{may2011}
and below
for how to do this.
\end{VerbatimOut}

\MyIO


% This list was made using data in
%     https://ctan.math.utah.edu/ctan/tex-archive/macros/latex/required/amscls/doc/amsthdoc.pdf
% which was used to make a list of definitions in PurdueThesis.cls
%     acknowledgment
%     assertion
%     assumption
%     axiom
%     case
%     claim
%     conclusion
%     condition
%     conjecture
%     corollary
%     criterion
%     definition
%     example
%     exercise
%     hypothesis
%     lemma
%     notation
%     note
%     observation (was defined in a earlier version of PurdueThesis)
%     problem
%     proof (is defined by ntheorem package)
%     property
%     proposition
%     question
%     remark
%     summary
%     theorem


\begin{VerbatimOut}{z.out}

By default all theorem-like constructs:\\[6pt]
\begin{tabular}{@{\hspace*{2\parindent}}llllll@{}}
acknowledgment& assertion&  assumption& axiom&      case&        claim\\
conclusion&     condition&  conjecture& corollary&  criterion&   definition\\
example&        exercise&   hypothesis& lemma&      notation&    note\\
observation&    problem&    proof&      property&   proposition& question\\
remark&         summary&    theorem\\
\end{tabular}
\mbox{}\\[6pt]
use the same counter.
\end{VerbatimOut}

\MyIO


\begin{VerbatimOut}{z.out}

From PurdueThesis.cls:
\begin{verbatim}
    \usepackage[amsthm]{ntheorem}
      % From example 31 of https://math.mit.edu/~poonen/papers/writing.pdf
      %     Use a single numbering system for all theorems, lemmas, etc.,
      %     instead of having both a Theorem 1.1 and a Lemma 1.1 in the
      %     same paper. This makes statements easier to find[...]
      % For example,
      %     \newtheorem{lemma}[theorem]{Lemma}
      % makes lemma use the theorem counter.
      \theoremstyle{plain}
      \newtheorem{theorem}{Theorem}[section]
      \newtheorem{lemma}[theorem]{Lemma}
      [28 more lines]
\end{verbatim}
To make,
for example,
lemma have its own counter put
\begin{verbatim}
    \renewtheorem{lemma}{Lemma}[section]
\end{verbatim}
in your thesis.tex file right after the |\begin{document}| statement.
\end{VerbatimOut}

\MyIO


\begin{VerbatimOut}{z.out}

\begin{acknowledgment}
  \MyRepeat{This is a sentence.  }{5}
\end{acknowledgment}
\index{\verb+\begin{acknowledgment}+}
\index{acknowledgment environment}
\end{VerbatimOut}

\MyIO


\begin{VerbatimOut}{z.out}

\begin{assertion}
  \MyRepeat{This is a sentence.  }{5}
\end{assertion}
\index{\verb+\begin{assertion}+}
\index{assertion environment}
\end{VerbatimOut}

\MyIO


\begin{VerbatimOut}{z.out}

\begin{assumption}
  \MyRepeat{This is a sentence.  }{5}
\end{assumption}
\index{\verb+\begin{assumption}+}
\index{assumption environment}
\end{VerbatimOut}

\MyIO


\begin{VerbatimOut}{z.out}

\begin{axiom}
  \MyRepeat{This is a sentence.  }{5}
\end{axiom}
\index{\verb+\begin{axiom}+}
\index{axiom environment}
\end{VerbatimOut}

\MyIO


% \begin{VerbatimOut}{z.out}
%
% \begin{case}
%   \MyRepeat{This is a sentence.  }
% \end{case}
% \index{\verb+\begin{case}+}
% \index{case environment}
% \end{VerbatimOut}
%
% \MyIO


\begin{VerbatimOut}{z.out}

\begin{claim}
  \MyRepeat{This is a sentence.  }{5}
\end{claim}
\index{\verb+\begin{claim}+}
\index{claim}
\end{VerbatimOut}

\MyIO


\begin{VerbatimOut}{z.out}

\begin{conclusion}
  \MyRepeat{This is a sentence.  }{5}
\end{conclusion}
\index{\verb+\begin{conclusion}+}
\index{conclusion environment}
\end{VerbatimOut}

\MyIO


\begin{VerbatimOut}{z.out}

\begin{condition}
  \MyRepeat{This is a sentence.  }{5}
\end{condition}
\index{\verb+\begin{condition}+}
\index{condition environment}
\end{VerbatimOut}

\MyIO


\begin{VerbatimOut}{z.out}

\begin{conjecture}
  \MyRepeat{This is a sentence.  }{5}
\end{conjecture}
\index{\verb+\begin{conjecture}+}
\index{conjecture environment}
\end{VerbatimOut}

\MyIO


\begin{VerbatimOut}{z.out}

\begin{corollary}
  \MyRepeat{This is a sentence.  }{5}
\end{corollary}
\index{\verb+\begin{corollary}+}
\index{corollary environment}
\end{VerbatimOut}

\MyIO


\begin{VerbatimOut}{z.out}

\begin{criterion}
  \MyRepeat{This is a sentence.  }{5}
\end{criterion}
\index{\verb+\begin{criterion}+}
\index{criterion environment}
\end{VerbatimOut}

\MyIO


\begin{VerbatimOut}{z.out}

\begin{definition}
  \MyRepeat{This is a sentence.  }{5}
\end{definition}
\index{\verb+\begin{definition}+}
\index{definition environment}
\end{VerbatimOut}

\MyIO


\begin{VerbatimOut}{z.out}

\begin{example}
  \MyRepeat{This is a sentence.  }{5}
\end{example}
\index{\verb+\begin{example}+}
\index{example environment}
\end{VerbatimOut}

\MyIO


\begin{VerbatimOut}{z.out}

\begin{exercise}
  \MyRepeat{This is a sentence.  }{5}
\end{exercise}
\index{\verb+\begin{exercise}+}
\index{exercise environment}
\end{VerbatimOut}

\MyIO


\begin{VerbatimOut}{z.out}

\begin{hypothesis}
  \MyRepeat{This is a sentence.  }{5}
\end{hypothesis}
\index{\verb+\begin{hypothesis}+}
\index{hypothesis environment}
\end{VerbatimOut}

\MyIO


\begin{VerbatimOut}{z.out}

\begin{lemma}
  \MyRepeat{This is a sentence.  }{5}
\end{lemma}
\index{\verb+\begin{lemma}+}
\index{lemma environment}
\end{VerbatimOut}

\MyIO


\begin{VerbatimOut}{z.out}

\begin{notation}
  \MyRepeat{This is a sentence.  }{5}
\end{notation}
\index{\verb+\begin{notation}+}
\index{notation environment}
\end{VerbatimOut}

\MyIO


\begin{VerbatimOut}{z.out}

\begin{note}
  \MyRepeat{This is a sentence.  }{5}
\end{note}
\index{\verb+\begin{note}+}
\index{note environment}
\end{VerbatimOut}

\MyIO


% \begin{VerbatimOut}{z.out}
%
% \begin{observation}
%   \MyRepeat{This is an example observation.  }{5}
% \end{observation}
% \index{\verb+\begin{observation}+}
% \index{observation environment}
% \end{VerbatimOut}
%
% \MyIO


\begin{VerbatimOut}{z.out}

\begin{problem}
  \MyRepeat{This is a sentence.  }{5}
\end{problem}
\index{\verb+\begin{problem}+}
\index{problem environment}
\end{VerbatimOut}

\MyIO


\begin{VerbatimOut}{z.out}

\begin{proof}
  \MyRepeat{This is an example proof.  }{5}
\end{proof}
\index{\verb+\begin{proof}+}
\index{proof environment}
\end{VerbatimOut}

\MyIO


\begin{VerbatimOut}{z.out}

\begin{property}
  \MyRepeat{This is a sentence.  }{5}
\end{property}
\index{\verb+\begin{property}+}
\index{property environment}
\end{VerbatimOut}

\MyIO


\begin{VerbatimOut}{z.out}

\begin{proposition}
  \MyRepeat{This is an example proposition.  }{5}
\end{proposition}
\index{\verb+\begin{proposition}+}
\index{proposition environment}
\end{VerbatimOut}

\MyIO


\begin{VerbatimOut}{z.out}

\begin{question}
  \MyRepeat{This is a sentence.  }{5}
\end{question}
\index{\verb+\begin{question}+}
\index{question environment}
\end{VerbatimOut}

\MyIO


\begin{VerbatimOut}{z.out}

\begin{remark}
  \MyRepeat{This is a sentence.  }{5}
\end{remark}
\index{\verb+\begin{remark}+}
\index{remark environment}
\end{VerbatimOut}

\MyIO


% There is already a \rule command.
% \begin{VerbatimOut}{z.out}
%
% \begin{rule}
%   This is an example rule.
%   This is an example rule.
%   This is an example rule.
%   This is an example rule.
%   This is an example rule.
% \end{rule}
% \index{\verb+\begin{rule}+}
% \index{rule environment}
% \end{VerbatimOut}
%
% \MyIO


% \begin{VerbatimOut}{z.out}
%
% \begin{solution}
%   \MyRepeat{This is an example solution.  }{5}
% \end{solution}
% \index{\verb+\begin{solution}+}
% \index{solution environment}
% \end{VerbatimOut}
%
% \MyIO


\begin{VerbatimOut}{z.out}

\begin{summary}
  \MyRepeat{This is a sentence.  }{5}
\end{summary}
\index{\verb+\begin{summary}+}
\index{summary environment}
\end{VerbatimOut}

\MyIO


\begin{VerbatimOut}{z.out}

\begin{theorem}
  \MyRepeat{This is an example theorem.  }{5}
\end{theorem}
\index{\verb+\begin{theorem}+}
\index{theorem environment}
\end{VerbatimOut}

\MyIO


\begin{VerbatimOut}{z.out}


\section{Examples}

\subsection{Bayes' Theorem}
\ix{Bayes' Theorem}
\ix{Bayes, Thomas}

Bayes' Theorem
\cite{bayes}
is
{
  \UndefineShortVerb{\|}
  \begin{equation}
    \text{P}(\text A|\text B)
    % The "\," puts a thin horizontal space there, 1/6 of an "em".
    % An "em" is roughly the width of a lowercase "m".
    = \frac{\text P(\text B|\text A)\,\text P(\text A)}{\text P(\text B)}
  \end{equation}
}
\end{VerbatimOut}

\MyIO


\begin{VerbatimOut}{z.out}

\subsection{Euler's identity}
\ix{Euler's identity}
\ix{Euler, Leonhard}

Euler's identity
\cite{eulers-identity}
is
\begin{equation}
  e^{i\pi} + 1 = 0.
\end{equation}
\end{VerbatimOut}

\MyIO


\begin{VerbatimOut}{z.out}

\subsection{Fourier transform}
\ix{Fourier Transform}

ISO 80000-2
\cite[page 25]{iso80000-2}
recommends using
\(\Fourier\)
nfor the Fourier transform.
The Fourier transform of \(f\) is
\(
  \displaystyle
  (\Fourier f)(\omega)
  = \Fourier (\omega)
  = \int_{-\infty}^\infty e^{i\omega t} f(t) \di t
\).
\end{VerbatimOut}

\MyIO


\begin{VerbatimOut}{z.out}

\subsection{Laplace transform}
\ix{Laplace Transform}

ISO 80000-2
\cite[page 25]{iso80000-2}
recommends using
\(\Laplace\)
for the Laplace transform.
The Laplace transform of \(s\) is
\(
  \displaystyle
  (\Laplace f)(s)
  = \Laplace(s)
  = \int_0^\infty e^{-st} f(t) \di t
\).
\end{VerbatimOut}

\MyIO


\begin{VerbatimOut}{z.out}

\subsection{Nicomachus's theorem}
\ix{Nicomachus's theorem}
\ix{Nicomachus of Gerasa}
Nicomachus's theorem
\cite{wikipedia-nicomachus}
states that
the sum of the first~\(n\) cubes is the square of the~\(n\)th triangular number.
That is,
\begin{equation}
  1^3 + 2^3 + 3^3 + \cdots + n^3 = (1 + 2 + 3 + \cdots + n)^2.
\end{equation}
The same equation may be written more compactly using the mathematical notation for summation:
\begin{equation*}
  \sum_{k=1}^n k^3 = \left(\sum_{k=1}^n k\right)^2.
\end{equation*}
Also see the diagram on that web page.
\end{VerbatimOut}

\MyIO


\begin{VerbatimOut}{z.out}

\subsection{Prime Number Theorem}
\ix{Prime Number Theorem}

\textcite{li2013}
\ix{Li, Henry}
suggested using a functional equation
from the Prime Number Theorem proof
as an example:
\begin{equation}
  \int_1^x
    \sum_{p\le u}
    \left\lfloor\frac{\log u}{\log p}\right\rfloor
    \log p
    \di u
    =
    \frac1{2\pi i}
    \int_{c-i\infty}^{c+i\infty}
    \frac{x^{s+1}}{s(s+1)}
    \left(-\frac{\zeta'(s)}{\zeta(s)}\right)
    \di s
\end{equation}
\end{VerbatimOut}

\MyIO


\begin{VerbatimOut}{z.out}

\subsection{Quantum Mechanics}
\ix{Quantum Mechanics}

\textcite{greene-2021-04-04}
wrote
\ix{Greene, Brian Randolph}
\begin{quotation}
  Quantum Mechanics in a nutshell:
  A particle goes from here to there
  by sampling every possible trajectory from here to there.

  \[
    \langle x_f,t_f \mid x,t_{\mathit i} \rangle
    =
    \sum_{\text p \in \text{paths}} e^{\mathit iS(\text p) \si{\planckbar}}
  \]
\end{quotation}
\end{VerbatimOut}

\MyIO


%  https://tex.stackexchange.com/questions/96568/how-can-i-align-multiple-cases-environment-simultaneously
\begin{VerbatimOut}{z.out}

\subsection{Question in String Theory / Mass of States / Number Operator}

\textcite{yourlazyphysicist2017}
wrote
``I have the following definition of the space-time coordinates'':
{
  % The following five definitions are local to inside the { ... }.
  \newcommand{\fpt}{{4\pi T}}
  \newcommand{\oh}{\frac12}
  \newcommand{\snnz}{\sum_{n\ne0}}
  \newcommand{\tms}{\tau - \sigma}
  \newcommand{\tps}{\tau + \sigma}
  \begin{align}
    \text{closed string: }&
      \begin{cases}
        \displaystyle
        X^\mu_R
          = \oh x^\mu
          + \frac1\fpt (\tms) p^\mu
          + \frac i{\sqrt\fpt} \snnz \frac1n \alpha^\mu_n e^{-in(\tms)},\\
        \displaystyle
        X^\mu_L
          = \oh x^\mu
          + \frac1\fpt (\tps) p^\mu
          + \frac i{\sqrt\fpt} \snnz \frac1n \tilde\alpha^\mu_n e^{-in(\tps)}.
      \end{cases}\\[6pt]
    \text{open string: }&
      \begin{cases}
        \displaystyle
        X^\mu_N
          = x^\mu
          + \frac1{\pi T}p^\mu\tau
          + \frac i{\sqrt{\pi T}} \snnz \frac1n \alpha^\mu_n e^{-in\tau} \cos(n\sigma),\\
        \displaystyle
        X^\mu_D
          = x^\mu
          + \frac i{\sqrt{\pi T}} \snnz \frac1n \alpha^\mu_n e^{-in\tau} \sin(n\sigma).
      \end{cases}
  \end{align}
}
\end{VerbatimOut}

\MyIO


\begin{VerbatimOut}{z.out}

\subsection{Willans' Formula}

\citetitle{rowland2022}
\cite{rowland2022}
\ix{Willans, C. P.}
\ix{Rowland, Eric Samuel}
is a good video about Willans' Formula by Eric Rowland:

\begin{equation}
  % Typeset i and j in a math italic font for this equation only.
  \MathIt{i}
  \MathIt{j}
  \text{\(n\)th prime}
  =
  1
  +
  \sum_{i = 1}^{2^n}
  \left\lfloor
    \left(
      \frac
        n
        {
          \displaystyle  % so \sum limits are above and below and to make denominator bigger
          \sum_{j=1}^i
          \left\lfloor
            \left(
              % use \, for a tiny bit of horizontal space
              \cos \pi \, \frac {(j-1)! + 1} j
            \right)^2
          \right\rfloor
        }
    \right)^{1/n}
  \right\rfloor
\end{equation}
\end{VerbatimOut}

\MyIO




% https://www.google.com/url?sa=i&rct=j&q=&esrc=s&source=images&cd=&cad=rja&uact=8&ved=0ahUKEwiSzf370-jmAhXIGs0KHRHyAZQQMwiCASgNMA0&url=https%3A%2F%2Fwww.chegg.com%2Fhomework-help%2Fquestions-and-answers%2Fequations-displacement-atoms-along-linear-chain-d2u-u-m-mass-atoms-c-force-constant-neares-q26052186&psig=AOvVaw1eYq-Q0El0Tbjgbp_Lu2Vv&ust=1578183004388936&ictx=3&uact=3

% M\frac{d^2\mu_n}{dt^2} = C(u_{n_1} - 2u_n + u_{n-1}).

% a^{b} same as a^b     c^{de} f^{g^h} ij^{kl}^{mn}

% superscripts
% subscripts

% operators

%\sum
%\prod

%\alpha
%\beta
%\gamma
%\delta
%\epsilor
%\varepsilon
%\zero
%\eta
%\theta
%\vartheta
%\iota
%\kappa
%\lambda
%\mu
%\nu\xi
%\varrho
%\sigma
%\varsigma
%\tau
%\upsilan
%\pha
%\varphi
%\chi
%\psi
%\omega

%\Gamma
%\Delta
%\Theta
%\Lambda
%\Xi
%\Pi
%\Sigmo
%\Upsilan
%\Phi
%\Psi
%\Omega

%mixed greek and text

%x_min, x_max

%\ldots

%\to

%1 + \frac14 + \frac19 + \cdots = \frac\pi6

%fractions

% frac{\ln x}{\ln\alpha} = log_\alpha x

% \sum_{i=1}^n = 1 + 2 + \cdots + n = \frac{n(n+1)}{2}


% xxxx     \end{verbatim}
% xxxx % Requires \usepackage{amsmath}; use align* for no equation number.
% xxxx \begin{align}
% xxxx   a = {}& b + c\\
% xxxx   x = {}& y + z
% xxxx \end{align}
% xxxx     \vskip\baselineskip
% xxxx     \hrule
% xxxx     \vskip0.5\baselineskip
% xxxx     \filbreak
% xxxx
% xxxx     \begin{verbatim}
% xxxx \[
% xxxx   Z =
% xxxx     \left(
% xxxx       \begin{array}{cc}
% xxxx         a& b\\
% xxxx         c& d
% xxxx       \end{array}
% xxxx     \right)
% xxxx \]
% xxxx     \end{verbatim}
% xxxx \[
% xxxx   Z =
% xxxx     \left(
% xxxx       \begin{array}{cc}
% xxxx         a& b\\
% xxxx         c& d
% xxxx       \end{array}
% xxxx     \right)
% xxxx \]
% xxxx     \vskip\baselineskip
% xxxx     \hrule
% xxxx     \vskip0.5\baselineskip
% xxxx     \filbreak
% xxxx
% xxxx     \begin{verbatim}
% xxxx \begin{equation}
% xxxx   \begin{split}
% xxxx     a = {}& b + c\\
% xxxx       & {} + d + e
% xxxx   \end{split}
% xxxx \end{equation}
% xxxx     \end{verbatim}
% xxxx \begin{equation}
% xxxx   \begin{split}
% xxxx     a = {}& b + c\\
% xxxx       & {} + d + e
% xxxx     \end{split}
% xxxx \end{equation}
% xxxx     \vskip\baselineskip
% xxxx     \hrule
% xxxx     \vskip0.5\baselineskip
% xxxx     \filbreak
% xxxx
% xxxx     \begin{verbatim}
% xxxx \[
% xxxx   (\cos x)^2 + (\sin x)^2 = 1
% xxxx \]
% xxxx     \end{verbatim}
% xxxx \[
% xxxx   (\cos x)^2 + (\sin x)^2 = 1
% xxxx \]
% xxxx     \vskip\baselineskip
% xxxx     \hrule
% xxxx     \vskip0.5\baselineskip
% xxxx     \filbreak
% xxxx
% xxxx     \begin{verbatim}
% xxxx If $X = \cos x$ and $Y = \sin x$ then $X^2 + Y^2 = 1$.
% xxxx     \end{verbatim}
% xxxx If $X = \cos x$ and $Y = \sin x$ then $X^2 + Y^2 = 1$.
% xxxx     \vskip\baselineskip
% xxxx     \hrule
% xxxx     \vskip0.5\baselineskip
% xxxx     \filbreak
% xxxx
% xxxx
